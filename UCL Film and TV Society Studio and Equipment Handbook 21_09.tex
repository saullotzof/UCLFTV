\documentclass[]{article}
\usepackage[subtle]{savetrees}
\usepackage{siunitx}
\usepackage{framed}
\usepackage[a4paper,margin=20mm]{geometry}
\usepackage{hyperref}
\hypersetup{
    colorlinks=true,
    linkcolor=blue,
    filecolor=black,      
    urlcolor=blue,
    citecolor=black
}
\urlstyle{same}
\begin{document}
\title{UCL Film and TV Society Studio and Equipment Handbook}
\author {Saul Lotzof, Çağrı Ustaoğlu, Harry Traherne and Jenny-Ann Windbrake}
\date{\today}
\maketitle
\tableofcontents
\section{Introduction}
Thank you for taking the time to read our Studio and Equipment Rules. These rules \textit{must} be followed when using any of UCL Film and TV Society's spaces or equipment.
\subsection{Contact Details}
\begin{itemize}
    \item Studio, Equipment and IT Manager
          \begin{itemize}
              \item Çağrı Ustaoğlu - \href{mailto:cagri.ustaoglu.19@ucl.ac.uk}{cagri.ustaoglu.19@ucl.ac.uk}
              \item Jenny-Ann Windbrake - \href{mailto:zctyjlw@ucl.ac.uk}{jenny-ann.windbrake.21@ucl.ac.uk}
              \item Harry Traherne - \href{mailto:harry.traherne.20@ucl.ac.uk}{harry.traherne.20@ucl.ac.uk}
          \end{itemize}
    \item Senior Committee
          \begin{itemize}
              \item Saul Lotzof - \href{mailto:saul.lotzof.20@ucl.ac.uk}{saul.lotzof.20@ucl.ac.uk}
              \item Jasmine Austin - \href{mailto:jasmine.austin.20@ucl.ac.uk}{jasmine.austin.20@ucl.ac.uk}
          \end{itemize}
\end{itemize}
\subsection{Definitions}
\begin{enumerate}
    \item ``Society'', ``FilmSoc'' refers to the UCL Film \& TV Society
    \item `'`SEIM'' refers to Studio, Equipment (and) IT Manager
    \item ``Senior Committee member'' refers to the President of the UCL Film \& TV Society, Saul Lotzof and Treasurer, Jasmine Austin
    \item ``Producers'' refers to the Drama Producer, Barbara Kononova or Documentary Producer, Nat Ng
    \item ``Hirer'' refers to the person leasing equipment from the UCL Film \& TV Society
\end{enumerate}
\subsection{Prerequisites}
Before you can book or use the studio and equipment, you \textit{must} be:
\begin{enumerate}
    \item A full `production' member of FilmSoc. You may be asked to provide your name and email address to us to cross-check your membership on our systems.
    \item Eligible to book studio/equipment (see \nameref{strikeSystem}).
\end{enumerate}
\section{Studio}
\subsection{The Space}\label{studioSpaces}
The studio is located in the Bloomsbury Theatre Building on the right side of the building. Entering via the Bloomsbury Theatre North entrance, the studio can be located at the end of the corridor and down the flight of stairs. Access may also be obtained via the lift. Studio access may only be achieved via an ID card of a committee member. The studio itself is divided up into three main sections, each of which can be booked individually.
\begin{enumerate}
    \item Main studio
    \item Sound recording room
    \item Post-production room
\end{enumerate}
\subsection{Rules on Bookings}
\begin{enumerate}
    \item Bookings must be made at least 3 days in advance of the start time of your booking.
    \item You may only make bookings for the studio up until the date that the current term ends, unless you are working on a term film, term documentary or affiliate project.
    \item Bookings cannot be made outside of UCL term dates. These can be found \href{https://www.ucl.ac.uk/students/life-ucl/term-dates-and-closures/term-dates-and-closures-2022-23}{here}. Bookings may not be made during College Reading Weeks.
    \item Bookings hours are dependent on the times that the Bloomsbury Theatre Building is open. This is typically 9:00 - 22:00 on weekdays and 9:00 - 18:00 on weekends. This may be subject to change and out of control of FilmSoc.
    \item Availability of studio space is at the discretion of the SEIMs and Senior Committee. A booking may be cancelled at any time by a member of the Senior Committee, or a SEIM. Some reasons for cancellation may include:
          \begin{itemize}
              \item Documentary or Term Film production precedence
              \item Consequence of strikes (see \nameref{strikeSystem})
              \item Incorrect booking details
          \end{itemize}
    \item As the sound recording room requires that activity in the post-production room be silent, concurrent bookings of the sound recording room and post-production room shall not be made. This is to ensure that adequate work may take place in each space. A SEIM may book both spaces, if the circumstances allow.
\end{enumerate}
\subsection{How to Make a Booking}
To make a booking, please first contact a SEIM. Your message must include the room you intend to book, and a few details of the project you will be working on. Once a SEIM has received your message and decided that the booking is appropriate, they will give you a permission code. With your permission code, you may now access the booking website, found \href{https://ucl-film-tv-society.booqable.shop/}{here}.

Once you have submitted your booking request through the website, a SEIM will confirm your booking via email within 3 days. If the above are not included in your enquiry, it will only delay confirming your booking.
\subsection{Studio use and etiquette}
\begin{enumerate}
    \item The studio is a shared space managed by FilmSoc. Please be considerate of other users in the space.
    \item This year, we plan to implement an access card system in conjunction with the Students' Union. Once your booking has been confirmed, details of how the process works will be shared with you by a SEIM.
    \item When your booking has finished, please ensure that you leave enough time to restore the studio back to a clean and orderly state. This includes stacking chairs and restoring furniture to its original position, sweeping the floor, packing away props and equipment etc. The studio \textit{must} be left as it was found.
    \item Only a SEIM or Senior Committee Member may assess whether the studio space has been left in a clean state.
    \item If there are any accidents, spills, or damage to the studio space, either a SEIM, Senior Committee Member or Producer \textit{must} be informed as soon as possible. We handle all incidents on a case-by-case basis according to Students' Union procedure.
\end{enumerate}
\section{Equipment}\label{equipment}
For production members of FilmSoc, equipment can be booked free of charge. Bookings cannot last for more than 1 week.

Equipment booking inquiries are subject to varying degrees of scrutiny. You may be asked to provide an overview as to why you require certain equipment. You may also be asked to provide a production schedule, if booking for longer periods of time. Equally, you are strongly encouraged to ask a SEIM for advice on which equipment they think would work best for your shoot or use case. 

Most equipment requires prior experience or training before use. If you require training, a SEIM will let you know about any upcoming sessions for such training, or assess your current skill level.
\subsection{Current list of equipment}
All Society equipment can be found on our \href{https://ucl-film-tv-society.booqable.shop/}{booking site}.
\subsection{Rules on Bookings}
\begin{enumerate}
    \item Booking must be made at least seven days in advance of the start time of your booking.
    \item You may only make bookings for equipment up until the date that the current term ends. Bookings for the next term shall open on the day that the previous term ends.
    \item Equipment rentals are at the discretion of the committee. A booking may be cancelled at any time by a member of the Senior Committee or a SEIM. A cancellation whilst kit is booked out will require that kit be returned within 24 hours (ideally ASAP). Reasons for cancellation may include:
          \begin{itemize}
              \item Documentary or Term Film production precedence
              \item Consequence of strikes (see \nameref{strikeSystem})
              \item Incorrect booking details
          \end{itemize}
    \item Bookings may be rejected if it is clear that the member booking has neglected to book all of the necessary equipment for their use case (for example, booking a camera without the right batteries or cables).
    \item Care of the equipment whilst booked out is the responsibility of the hirer. Kit \textit{must} be checked out by a SEIM or a member of the Senior Committee. This means that any strikes will apply only to you, regardless of who was looking after or using the equipment.
    \item Under no circumstances may any equipment be taken out of the storage cupboards that is not included on your booking, regardless of value or urgency.
    \item Kit being returned to the studio \textit{must} receive the approval of a SEIM or member of Senior Committee. This approval may be verbal (in-person) or as a text (if \textbf{all} the equipment is confirmed to be returned to the studio.) 
    \item When any kit is returned, it \textit{must} be placed at the back of the downstairs studio in an ordered pile, regardless of whether you have been using the studio or not. Under no circumstances must you attempt to put back any equipment yourself.
    \item Storage items should be given to you pre-formatted. If you find you have been given a card which has data on it, please format the card before you use it. Upon returning a card, please ensure that all important data has been copied by you. Upon check-in of a storage media, we will format that media, erasing any data that you might have on it. You may not keep the storage item beyond your allocated booking period. You may not store any of your data on the network-attached storage (NAS) for any period of time, unless working on a term film or documentary.
    \item If you have made a room booking, you \textit{must} take a video of the space before you leave it, and send it to a SEIM. The video can be short, but needs to show the entire space. It is the discretion of the Senior Committee and SEIMs as to whether a video has not adequately shown the space, if there is a dispute about how the space has been left.
\end{enumerate}
\subsection{How to Make a Booking}
To make a booking, please first contact a SEIM. Your message \textit{must} include a rough list of the equipment you intend to borrow, and details of your experience with all of the equipment you request. Once a SEIM has received your message and decided that the booking is appropriate, they will give you a permission code. With your permission code, you may now access the booking website, found \href{https://ucl-film-tv-society.booqable.shop/}{here}. 

Please note that we advise be made two weeks prior to the intended borrowing dates. Your booking will be confirmed within seven days. If you wish to hire equipment in less than seven days time, please contact a SEIM directly, who will endeavour to assist you. They cannot guarantee that your booking will be made. If the above are not included in your enquiry, it will only delay confirming your booking.
\subsection{Equipment Use and Etiquette}
FilmSoc is the owner of all property mentioned above and is subject to UCL Students' Union inventory policy. All equipment is insured and kept secured in the FilmSoc storage facilities.
\begin{enumerate}
    \item Any incidents/damage to equipment \textit{must} be reported ASAP. Damage assessment and repair cost quote is the sole responsibility of the SEIM. No attempts to repair the equipment should be made by anyone other than those who receive the express approval of a SEIM.
    \item All damages are assessed on a case-by-case basis. For small damages, you may be billed directly. For larger damages, insurance policies protect you and our equipment.
    \item You \textit{must} return all equipment punctually. SEIMs will wait for 15 minutes without prior warning of delay, after which point they will leave and you will be given a strike.
    \item You \textit{must} return all equipment \textit{exactly} how it was given to you. This includes coiling cables, packing equipment in the appropriate manner and mainting or replacing bubblewrap used for safe protection. You are encouraged to ask a SEIM for more bubblewrap if you think you may need more.
\end{enumerate}
\section{Strike system}\label{strikeSystem}
To ensure smooth operation, we have a strike system in place. For breaching the rules aforementioned, one may receive a `strike'. Strikes impact whether studio time or equipment is given. Before a strike is given, a discussion between all SEIMs and Senior Committee will be had to ensure fair usage of this system.

If one is to receive three strikes, they will be barred from all space and equipment bookings. Receiving three strikes will not be considered as grounds to approve a request to refund FilmSoc production membership.
\appendix
\section{Example Studio Booking Message}\label{studioForm}
\begin{framed}
    \noindent \textbf{Subject: Request to book FilmSoc Equipment.} \newline
    \newline
    \noindent
    Hey! My name is Stan Joseph. I'm looking to book out the Blackmagic 4K, Samyang Cine Primes, Shoulder Rig Kit, Manfrotto 504HD, and an SSD. I used this camera and these lenses when I helped with filming Varsity last year, so I know how to use them. I'd be really grateful if you could send me a permission code if this sounds alright. Thank you!
\end{framed}
\end{document}